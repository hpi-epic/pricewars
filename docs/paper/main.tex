% This is "sig-alternate.tex" V2.1 April 2013

\documentclass{sig-alternate-05-2015}

%----------------------------------------------------------------------------------------
%	PACKAGES AND DOCUMENT CONFIGURATIONS
%----------------------------------------------------------------------------------------
%
% please be aware to install sudo apt-get install texlive-science
%
%
\usepackage{makeidx}                    % allows for indexgeneration
\usepackage{xargs}                      % Use more than one optional parameter in a new commands
\usepackage[pdftex,dvipsnames,table]{xcolor}  % Coloured text etc.
% \usepackage{hyperref}
\usepackage{graphicx} 
\usepackage{enumitem}
\usepackage{scrextend}
\usepackage{pdfpages}
\usepackage{tabularx}
\usepackage{multirow}
\usepackage{array}

\newcommand{\specialcell}[2][c]{%
  \begin{tabular}[#1]{@{}c@{}}#2\end{tabular}}
  
% takes the merchant letter as first parameter and the percentage of higher profit as second parameter
\newcommand{\evalresult}[2]{\specialcell{#1\\[-1ex]{\fontsize{7pt}{0pt}\selectfont+#2\%}}}

\usepackage{enumitem}
\setlist{noitemsep}

\usepackage{subcaption}

\usepackage[utf8]{inputenc}
\usepackage[T1]{fontenc}

\usepackage{scrextend} %for refnote
\usepackage{tabu}

\usepackage{eurosym}

\usepackage{url}
\usepackage{breakurl}
\usepackage[breaklinks]{hyperref} % cleverref
\usepackage[capitalise]{cleveref}
\def\UrlBreaks{\do\/\do-}

% 
\usepackage[colorinlistoftodos,prependcaption,textsize=tiny]{todonotes}
\newcommandx{\unsure}[2][1=]{\todo[linecolor=red,backgroundcolor=red!25,bordercolor=red,#1]{#2}}
\newcommandx{\change}[2][1=]{\todo[linecolor=blue,backgroundcolor=blue!25,bordercolor=blue,#1]{#2}}
\newcommandx{\info}[2][1=]{\todo[linecolor=OliveGreen,backgroundcolor=OliveGreen!25,bordercolor=OliveGreen,#1]{#2}}
\newcommandx{\improvement}[2][1=]{\todo[linecolor=Plum,backgroundcolor=Plum!25,bordercolor=Plum,#1]{#2}}
\newcommandx{\thiswillnotshow}[2][1=]{\todo[disable,#1]{#2}}
%
\setlength{\parindent}{0pt}

\usepackage[numbers]{natbib}

\widowpenalty=10000
\raggedbottom
\parfillskip 0pt plus 0.75\textwidth

% custom todo
%\newcommand{\todo}[1]{}
%\renewcommand{\todo}[1]{{\color{red} TODO: {#1}}}

%% DEBUG
%\usepackage{showframe}

\begin{document}

% Copyright
%\setcopyright{acmcopyright}
%\setcopyright{acmlicensed}
%\setcopyright{rightsretained}
%\setcopyright{usgov}
%\setcopyright{usgovmixed}
%\setcopyright{cagov}
%\setcopyright{cagovmixed}



%----------------------------------------------------------------------------------------
%	Meta Data
%----------------------------------------------------------------------------------------
%


\title{Simulating Strategic Interaction on Online Marketplaces}
\subtitle{[How to Survive Dynamic Pricing Competition]}



\numberofauthors{9} %  in this sample file, there are a *total*
% of EIGHT authors. SIX appear on the 'first-page' (for formatting
% reasons) and the remaining two appear in the \additionalauthors section.
%
\author{Marvin Bornstein, Johanna Latt, Jan Lindemann, Nikolai J. Podlesny, Sebastian Serth, Jan Selke}
\additionalauthors{Martin Boissier, Rainer Schlosser, Matthias Uflacker}
    

\maketitle

%
%----------------------------------------------------------------------------------------
%	Content
%----------------------------------------------------------------------------------------
%
%%%%%%%%%%%%%%%%%%%%%%%%%%     Abstract     %%%%%%%%%%%%%%%%%%%%%%%%%% 
%
%

\begin{abstract}
Implementing intelligent and efficient pricing strategies for big online marketplaces to make customers buy your product instead of the competitor's is one of the biggest challenges retailers have nowadays. Especially since pricing algorithms can vary heavily from simply rule-based to complex, data-driven strategies. Currently, retailers lack the possibility to test their algorithms appropriately before releasing them into the real world. Additionally, it is hard for researchers to investigate how pricing strategies react and interact based on influences and against each other. We picked up the challenge to create an environment to imitate different market situations in eCommerce. Therefore, we provide the toolkit to simulate and test pricing strategies in various contexts and with different consumer and merchant behaviors on such a marketplace.
\end{abstract}


%\keywords{Dynamic Pricing}

%\todo[inline]{MB: finde das Paper soweit echt gut. Aufbau ist sinnig, Motivation kommt langsam gut rüber. Ein wenig fehlt mir noch der rote Pfaden (ist aber nicht schlimm, wenn der nicht sitzt, andere Dinge sind wichtiger). Hauptaugenmerk ist in meinen Augen jetzt die Merchant-Vergleiche sauber in das Paper zu bekommen (Sec. 8).}

%
\section{Introduction}
\label{sec:Introduction}
% Owner: Jani
% Reviewed: 
%
While for the last couple of centuries prices were mainly defined through rule-based actions, more and more companies start basing their pricing calculations and strategies on technology and data-driven processes. Not only the actual computation of profit and cost is nowadays algorithm driven, but also its update and review strategy.
One of the most competitive and advanced fields is the algorithmic trading or high-frequency trading on stock exchanges. But also each one of us experiences nowadays technology driven price calculation on online marketplace like amazon, which we will hereinafter referrer to as dynamic pricing. 

Currently, pricing strategies and algorithms exist, but handlers of those mechanism lack of the possibility to test them appropriately before releasing them into the real world where they can create huge loses \citep{uflacker2016ertragsmanagement} \citep{schlosser2016optimal} \citep{schlosser2016stochastic} \citep{schlosser2016survive}. We picked up the challenge to create such an environment, imitating different market situations and therefore testing how pricing strategies react and interact based on influences and against each other.\\

\textbf{Contribution:} In this work, we briefly elaborate our process of building a distributed and scalable platform to imitate market situations and simulate dynamic pricing algorithms and their effects with potential real world settings.
Therefore, the following Chapter \ref{sec:Architecture} contains a short introduction into the underlying architecture of the platform.
The choreography of different services is described in Chapter \ref{sec:Choreography}. Chapter \ref{sec:Behaviors} will provide insights in the already implemented algorithms and their behaviors.
A user facing interface on top of the RESTful API used for the communication between the services is delineated in Chapter \ref{sec:ui} and finally Chapter \ref{sec:conclusion} concludes this elaboration. \\

%
% TODO: Do we want to keep this section as standalone? And if so, we need to include it in Contribution overview of the chapters. 
%
\section{Design Goals}
\label{sec:Design_Goals}
% Owner: Johanna
% Reviewed: 
%
While building the simulation platform, we had a set of specific design goals and restrictions in mind, that we wanted to fulfill to make the platform and the resulting simulation as realistic, dynamic and reactive as possible: 
\begin{itemize}
    \item \textbf{Reactivity:} Allow pricing algorithms to do fully dynamic price updates and look-ups at any time without being refrained by discrete time intervals to enable truly reactive pricing strategies.
    \item \textbf{Security:} Prevent fraud in simulation setups with multiple participants to allow the simulation to be used for competitions or education purposes. Fraud could be the deliberate influence on the expenses or profits of other pricing algorithms or the usage of data (e.g. for learning purposes) that is normally not accessible.
    \item \textbf{Realism:} Enforce real-life restrictions present in every big real-life marketplaces such as a limited amount of price updates per time interval to avoid advantages by constantly changing prices.
    \item \textbf{Flexibility:} Offer the greatest possible flexibility in terms of scalability and adaptivity of the system to make the system easily expandable and adaptable to new possible design goals or user needs. 
\end{itemize}

%Check intro in \url{http://faculty.chicagobooth.edu/workshops/omscience/pdf/Spring%202016/Popescu.pdf}

%
\section{Related Work}
\label{sec:Related_Work}
% Owner: 
% Reviewed: 
%
(only own section if we find enough) \\

\begin{itemize}
    \item Bertrand (1883) model: Exact same product, two companies -> product will be offered at purchase price by both because costumers only have one differentiating feature: the price. so they buy the cheapest, no room for profit.
    \item 
\end{itemize}

Check literature review in \url{http://faculty.chicagobooth.edu/workshops/omscience/pdf/Spring%202016/Popescu.pdf}
\section{Architecture}
\label{sec:Architecture}
% Owner: Jani
% Reviewed:
%
Confronted with the challenge of creating a high-performance and expandable infrastructure for simulating a marketplace with different merchants and consumers, a microservice architecture was created allowing the user to scale, exchange or add single service ad-hoc and on demand. Each service within our architecture implements one business artifact. This architecture pattern comes with the cost of a communication overhead and requires farsighted API design.\\

Figure~\ref{fig:fmc} describes the underlying architecture modeling as FMC\footnote{\url{http://www.fmc-modeling.org/}} diagram. When initiating a new simulation universe, it comes along with the marketplace component, as well as the producer and a management ui for controlling each service. While the producer offers products, the marketplace holds the current market situation and handles price updates and purchases of goods. Each transaction handled by the producer and marketplace is logged to a stream database, namely Apache Kafka\footnote{\url{https://kafka.apache.org/}}. Further, those logs are being analyzed and aggregated through a batch data processing component which is in our case Apache Flink\footnote{\url{https://flink.apache.org/}} and written back into a new Kafka topic. Those details, then, can be accessed through a socket connection or REST interface provided by a kafka-reverse-proxy. 
To liven the place up, numerous consumers or merchants may join and participate in this market simulation. By default, one consumer and five\todo{are it really five?} merchants are deployed with already implemented strategy. Those behaviors are deepened in chapter \ref{sec:Behaviors} while the choreography of the single services is delineated in chapter \ref{sec:Choreography}.

%
\begin{figure}[h]
    \centering
    \includegraphics[width=0.5\textwidth]{images/architecture_fmc.png}
    \caption{FMC diagram of the Price Wars architecture}
    \label{fig:fmc}
\end{figure}
%
\section{Service Choreography}
\label{sec:Choreography}
% Owner: 
% Reviewed:
@TODO: how do the services interact, how do we secure some major challenges in short sentences.

No ticks or such, but completely free and dynamic, every merchant can check or update prices at any time -> close to real life (unlike eg http://www.informsrmp2017.com/description-challenge.pdf)

\begin{itemize}
\item event logs, kafka etc
\item fraud / cheating
\item (inter service communication (via REST and connection pools))
\item (where are limits / bottlenecks?)
\end{itemize} 

\section{Behaviors}
\label{sec:Behaviors}
% Owner: 
% Reviewed:
Consumer:
\begin{itemize}
\item Sigmoid distribution around twice of producer price
\item Logistic regression coefficients which are used to calculate selling probability for consumer to buy
\end{itemize}

Merchant:
\begin{itemize}
\item Gas Station strategy
\item Logistic regression 
\item Be the n-cheapest
\item fix price
\end{itemize}
\section{User Interface}
\label{sec:ui}
% Owner: Jani, Johanna
% Reviewed: 
%
The ``Management User Interface'' (UI) enables the user to configure, operate and orchestrate the different microservices all in one place. 

Consuming the RESTful APIs exposed by each individual service, we use websockets to realize a a real-time streaming connection to the kafka-reverse-proxy component that provides the UI with all necessary data.

%Its implementation is based on angularJS consuming the RESTful APIs exposed by each individual service. Additionally, we used the socket.io\footnote{\url{https://github.com/socketio/socket.io}} technology which imitates a websocket connection based on HTTP to realize a real-time streaming connection to the kafka-reverse-proxy as consumer of the Apache Kafka\footnote{\url{https://kafka.apache.org/}} instances.

%Consuming the RESTful APIs exposed by each individual service, we additionally used websockets to realize a real-time streaming connection to the kafka-reverse-proxy as consumer of the Apache Kafka\footnote{\url{https://kafka.apache.org/}} instances.

The UI allows a user of the simulation to start and stop a simulation in the sense that all components having a state can be started, stopped, accessed and configured here, including their state. These components are the merchants and the consumer. Furthermore, all exposed settings for the behavior of each merchant and the consumer can be viewed, edited and updated.

The stateless components of the simulation, such as the producer and the marketplace, cannot be stopped or started, but configured here as well. E.g., products can be added or deleted, or the marketplace can be emptied with the necessary login-credentials for the underlying database.

If a user wants to register a new merchant and send it into the simulation, they can use the UI to register a new endpoint under which the merchant is running, and in return receive a secret token that is used for authorization (rf. \cref{sec:DecentralizedAuthorization}). 

Furthermore, the UI also offers easy access to visualizations of the real-time pricing interaction of merchants and to key performance indicators of each merchant, simplifying the process of comparing different pricing strategies.
%
\section{Evaluation}
\label{sec:evaluation}
% Owner: Jani
% Reviewed:
%
We used the platform built to evaluate the performance of the already provided merchant behaviors. Thus, we wanted to examine the quality of and the possibilities offered by the simulation. We also evaluated the amount of data generated by the simulation and tried to identify bottlenecks for the system's performance.

%Within the first simulation rounds, we already gained several interesting insights regarding the merchant performance and the system load needed to be handled. Additionally, the generated data sizes will be addressed.

\subsection{Merchant Performance}
\label{sec:merchant_evaluation}
% Owner: Jani
% Reviewed:
%
We assumed that the exemplary data-driven approach based on a logistic regression pricing model promises higher profits than common rule-based behaviors because of its better adjustment to the consumer behavior in pricing. To test this hypothesize, we performed a one-on-one evaluation of the 6 provided merchant behaviors from \cref{sec:Behaviors_Merchants}. We ran each evaluation with the following settings:

\begin{itemize}
    \item Duration: 20 minutes
    \item Consumer:
    \begin{itemize}[nosep]
        \item Max. 20 purchases per minute
        \item 100\% logit-behavior
    \end{itemize}
    \item Producer: 1 product with 4 qualities
    \item Merchant: 
    \begin{itemize}
        \item 3 products in stock per merchant
        \item 2 updates per second
    \end{itemize}
\end{itemize}

The results can be found in \cref{table:merchant-evaluation-matrix}.

{
\setlength\extrarowheight{2pt}
\begin{table}[ht]
\centering
\caption{Performance results of the one-on-one merchant behavior evaluation. The letter in each cell denotes the merchant of the pair that gained the higher profit, i.e. won the simulation. The number below the letter tells how much more profit than their competitor they gained. A: Cheapest, B: Second Cheapest, C: Random Thee, D: Two-Bound, E: Fixed Price, F: Machine-Learning }
\label{table:merchant-evaluation-matrix}
\begin{tabular}{|r||c|c|c|c|c|c|c|}
\hline
 & \textbf{A} & \textbf{B} & \textbf{C} & \textbf{D} & \textbf{E} & \textbf{F}  \\ \hline \hline
\textbf{A} & & \cellcolor{lightgray} & \cellcolor{lightgray} & \cellcolor{lightgray} & \cellcolor{lightgray} & \cellcolor{lightgray} \\ \hline
\textbf{B} & \evalresult{B}{76,16} & & \cellcolor{lightgray} & \cellcolor{lightgray} & \cellcolor{lightgray} & \cellcolor{lightgray} \\ \hline
\textbf{C} & \evalresult{A{76,16} & & & \cellcolor{lightgray} & \cellcolor{lightgray}& \cellcolor{lightgray} \\ \hline
\textbf{D} & & & & & \cellcolor{lightgray} & \cellcolor{lightgray} \\ \hline
\textbf{E} & & & & & & \cellcolor{lightgray} \\ \hline
\textbf{F} & & & \evalresult{F}{70} & & &  \\ \hline
\end{tabular}
\end{table}
}

We also ran an evaluation with all merchants running at once to see whether the data-driven merchant would beat the other, rule-based merchants. The same settings as above were used except for the consumer who bought a maximum of 60 items per minute to account for the increased number of merchants. The results can be seen in \cref{table:all-merchants-evaluation}.

{
\setlength\extrarowheight{2pt}
\begin{table}[ht]
\centering
\caption{Performance results of the all merchant evaluation. The numbers are the profit gained by each merchant after 20 minutes of simulation, i.e. the higher the better. A: Cheapest, B: Second Cheapest, C: Random Thee, D: Two-Bound, E: Fixed Price, F: Machine-Learning}
\label{table:all-merchants-evaluation}
\begin{tabular}{|c|c|c|c|c|c|c|}
\hline
\textbf{A} & \textbf{B} & \textbf{C} & \textbf{D} & \textbf{E} & \textbf{F}  \\ \hline \hline
 & & & & &  \\ \hline
\end{tabular}
\end{table}
}

The results clearly show that the data-driven merchant outperforms all other, rule-based merchants in a one-on-one setup. This is in line with the hypothesize we formulated above. However, the simulation with six merchants at once disproves this hypothesize - the data-driven merchant only had the second-best result and was beaten by the two-bound, rule-based strategy, that clearly lost against the data-driven merchant in the one-on-one simulation. 
This result shows very well how important and unpredictable the influence of interaction effects between multiple merchants can be on the results and that simply testing and evaluating strategies on a one-on-one basis is not sufficient to assess its performance.

Thus, these results demonstrate the usefulness and possible applications of the platform built. It allows to account for these unforeseeable interaction effects by enabling the user to run quick and complex simulations of possible real-world setups with an almost arbitrarily high variety of merchants and strategies. At the same time it offers comprehensive evaluation metrics, giving the user immediate feedback on the performance of each merchant, fulfilling many of the design goals we postulated at the beginning in \cref{sec:Design_Goals}. 

Of course, further tests have to be run with other data-driven strategies as well, especially to investigate how data-driven strategies perform when they play against each other and not only against rule-based strategies. 


%
\subsection{Produced Data}
% Owner: Johanna
% Reviewed:
%

We did an evaluation of the data size produced by a simulation using the following benchmarks:
\begin{itemize}
    \item Duration: 1 hour
    \item Products: 4000 with 4 different qualities each, i.e. 16,000 products in total
    \item Merchants: 5 merchants
    \begin{itemize}[nosep]
        \item 750 products in stock per merchant
        \item 17 requests per second per merchant
    \end{itemize}
    \item Consumer: Max. 1000 purchases per minute
\end{itemize}

The amount of data generated and persisted per service can be seen in \cref{table:generated_data}.

%\begin{table}[ht]
%\centering
%\caption{Data size generated by a 1-hour simulation with x merchants, 4000 products and max. 2000 actions per minute}
%\label{table:generated_data}
%\begin{tabular}{|l|r|}
%\hline
%\textbf{Service} & \textbf{Data Size} \\ \hline \hline
%Kafka            & 2.3 GB             \\ \hline
%Merchants (total)& 25 MB              \\ \hline
%Marketplace      & 24 MB              \\ \hline
%Producer         & 4 MB               \\ \hline
%UI               & 2 MB               \\ \hline
%Consumer         & 60 KB              \\ \hline
%\end{tabular}
%\end{table}

\begin{table}[ht]
\centering
\caption{Data size generated by a 1-hour simulation with 5 merchants and 4000 products.}
\label{table:generated_data}
\begin{tabular}{|l|r|}
\hline
\textbf{Service} & \textbf{Data Size} \\ \hline \hline
Kafka            & 173 MB             \\ \hline
Merchants (total)& 109 MB             \\ \hline
Marketplace      & 72 MB              \\ \hline
Producer         & 13 MB              \\ \hline
UI               & 1.7 MB             \\ \hline
Consumer         & 800 KB             \\ \hline
\end{tabular}
\end{table}

\todo[inline]{is this really data generated in one hour? I think it also includes data that is just always present, eg the 4 MB at the producer won’t change anymore, they are just due to the high number of products } 
\todo[inline]{we started with 7 merchants, one stopped at the beginning and two died later - do we say we tested it with 4, 5, 6 or 7 merchants...?}

\subsection{System Load}
\label{sec:system_evaluation}
% Owner: 
% Reviewed:
%
The developed system is able to handle very high loads in terms of the number of merchants, offers or products. The only potentially problematic bottlenecks are the user interface (see also \cref{sec:FutureWork}) and the amount of produced data as stated in the previous section, depending on the duration of the simulation and the underlying machine's capacities. 

As shown during the evaluation of the data sizes, a very high number of products is no problem for the system - only the creation of those 4000 products took some time, but the overall performance of the simulation itself was not affected negatively. 

The load on the marketplace is determined by the number of requests per minute that a merchant and the consumer can make. This number is adjustable to allow the simulation of times of higher or lower demand on the consumer side (such as Christmas) or to simply speed up the simulation if all parameters are increased equally. As implied above, we ran simulations with more than 100 requests per second without any problems. Also, simulations with 6 merchants at once, doing 2 updates per second each, and a consumer buying 60 products per minute, were absolutely no problem for the system. 

%\todo[inline]{kann jemand abschätzen, was das maximum an gesamt-requests ist, das wir bisher problemlos getestet haben/handeln könnten?}
%\todo[inline]{können wir ne abschätzung geben, wie viele merchants max laufen können (gegeben max anzahl von requests von x/minute)?}


\section{Future Work}
\label{sec:FutureWork}
% Owner: Jani, Johanna
% Reviewed: 
%
While providing an easy and comprehensive way to simulate different market situations based on a variety of consumer behaviors and competing merchants, the current solution still has a lot of potential for extension and improvement. 

In the current setup, the producer only offers goods without any expiration date. But when thinking about plane or festival tickets, perishables or any other short-life products, pricing strategies might perform very differently and have to adapt to completely new features. Thus, it would be interesting to add the possibility to offer such perishable products. In consequence, a more comprehensive notion of time would have to be introduced throughout the whole system, and the marketplace would have to check and verify a product's expiration date with the producer. Additionally, the behaviors of the consumer as well as within the merchants need to react on those additional attributes.

Another very interesting case to cover in the near future could be consumer ratings and how they influence pricing strategies. Having now such an environment to simulate different market situations and consumer demands, different consumer ratings may also influence pricing strategies significantly. 

Lastly, it would be interesting to extend the simulation to support buying strategies within merchants in addition to pricing strategies, meaning that merchants no longer receive a fixed number of random products, but can decide on what, when and how much they want to buy themselves. This would add a lot of additional complexity to the design of merchant strategies, but would be an even closer simulation of the real world.

We are also aware that the current solution is far from being perfect. Components of the current solution that need further revision in the future are, in particular, the UI and also the Kafka-related components.

The user interfaces turned out to be a bottleneck in the current setup when the simulation is under a high load, i.e. there are many price updates and sales. Especially the real-time price interaction graphs start to lag heavily or even crash the browser under high load, making parts of the UI almost unusable. To solve this problem, a complete refactoring of this UI component or a switch of the underlying third party charts library might be necessary.

The Kafka-components might need to be refactored in terms of the size of the produced data. As shown in \cref{sec:evaluation}, the amount of data produced in just one hour of simulation is above 2 GB. While that amount should be rather easy to handle on most systems, potential long term simulations - as might be needed for more sophisticated data-driven pricing strategies or long-term algorithm evaluations - might produce more problematic data amounts. A more considerate handling of outdated data as well as a stronger compression of current data might be needed. 


%OLD: We are aware that the current solution is far from being perfect. One is already able to simulate different market situations based on a variety of consumer behaviors and competing merchants, however, this solution can be extended to cover even more possibilities. Currently, the producer may provide goods without any expiration date. But when thinking about plane or festival tickets, perishables or any other short-life products, those can be also included in a later step as simulation content. If so, the producer may include an expiration date and optionally a cap of items (for the air plane case) which consequently has to be checked and verified by the marketplace. Additionally, the behaviors by the consumer as well as within the merchants may need to react one those additional attributes.\\

\section{Conclusion}
\label{sec:conclusion}
% Owner: Jani
% Reviewed:
%
In this work, we presented a distributed and scalable platform to imitate market situations and simulate dynamic pricing algorithms and their effects with potential real world settings. With this toolkit, handler may be enabled to evaluate their pricing strategies appropriately before releasing them into the real world where they can create huge loses. Simultaneously, students and researchers may now implement and evaluate how pricing strategies react and interact based on influences and against each other.\\

The source code and its technical documentation will be publicly available at github.com\footnote{\url{https://github.com/hpi-epic/masterproject-pricewars}}
while a screencast is accessible on YouTube\footnote{\url{https://www.youtube.com/watch?v=bqXSi5cv8cE}}.


%\end{document}  % This is where a 'short' article might terminate

%\newpage

%ACKNOWLEDGMENTS are optional
\section*{Acknowledgments}
As part of this elaboration, special thanks goes to Dr. Matthias Uflacker, Dr. Rainer Schlosser, and Martin Boissier for their continuous support and supervision. Also, we thank all attendant members of the EPIC research group for their fruitful discussions.


\bibliographystyle{unsrtnat}
\bibliography{bibtex/references.bib}  


\end{document}
